\chapter{EXEMPLES DE LISTES ET BOÎTES DE TEXTE}

% Exemple de liste en points
Exemple de liste en points
\begin{itemize}
  \item La liste commence par la fonction \verb|\begin{itemize}| et chaque objet (point) de la liste commence par la fonction \verb|\item|.
  \item Chaque objet de la liste commence par un point automatiquement placé.
  \item Le texte de chaque objet peut être de n'importe quelle taille.
  \item La liste doit OBLIGATOIREMENT se terminer par la fonction \verb|\end{itemize}|
\end{itemize}

% Exemple de liste ordonnée

Exemple de liste ordonnée
\begin{enumerate}
  \item La liste commence par la fonction \verb|\begin{enumerate}| et chaque objet (chiffre) de la liste commence par la fonction \verb|\item|.
  \item Les objets de la liste sont ordonnés et le chiffre est apposé automatiquement.
  \item Le texte de chaque objet peut être de n'importe quelle taille.
  \item La liste doit OBLIGATOIREMENT se terminer par la fonction \verb|\end{enumerate}|
\end{enumerate}

% Exemple de boîte de texte

Deux exemples de boîtes de texte :

Une boîte de texte avec fond gris/parchemin

\begin{textbox_parchemin}{Titre de la boîte de texte}
    Entrer le texte qui doit être dans la boîte ici. Cette longue ligne de texte permettra de vérifier l'interligne qui change à l'intérieur de la boîte, par rapport au corps du texte.
\end{textbox_parchemin}

Une boîte de texte avec fond blanc

\begin{textbox_blanc}{Titre de la boîte de texte}
    Entrer le texte qui doit être dans la boîte ici.
    \\
    \\
    La boîte supporte des sauts de ligne.
    \\
    \\
    \\
    \\
    Mais aussi.
    \\
    \\
    \\
    \\
    Les sauts de page.
    \\
    \\
    \\
    \\
    Sans briser la boîte.
\end{textbox_blanc}